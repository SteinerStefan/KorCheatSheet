\section{Basic Korean}
\red{Three Basic rules}\\
\impbox{\begin{enumerate}
\item \red{Every Korean sentence must end in either a verb or an adjective}
\item Every Korean verb and adjective ends with \begin{Korean}다\end{Korean} in the dictionary form
\item A lot of verbs and adjectives end with \begin{Korean}하다\end{Korean}. To get the noun of these words simply \begin{Korean}하다\end{Korean} has to be \red{removed}.\\
e.g.: \begin{Korean}
행복하다 = happy$\qquad$ 행복    = happyness
\end{Korean}
\end{enumerate}}
% % % % % % % % % % % % % %
%Ending with a verb
% % % % % % % % % % % % % %
\subsection{sentence ending with a verb}
The Structure of a sentence ending with a verb is: \\
\impbox{\subject{Subject} - \object{object} - \ver{verb}}

\subsubsection{subject marker}

\begin{tabular}{p{200pt}p{200pt}}
\hline
	 \textcolor{red}{subject} particle  	\begin{Korean} 은, 는\end{Korean}&
	
	Vowel:\begin{Korean} \red{는}: 나는, 자는 \end{Korean} 	\newline
	Consonant:\begin{Korean}\red{은}: 집은,잭은 \end{Korean}\\
	
\hline
	 \textcolor{red}{subject} paticle  	\begin{Korean} 이, 가\end{Korean}&
	
	Vowel:\begin{Korean} \red{가}: 고양이가 집 뒤에 있다 \end{Korean} 	\newline
	Consonant:\begin{Korean}\red{이}: 젝이 집 뒤에 있다 \end{Korean}\\
\hline
\end{tabular}\\
\\\\
\begin{Korean}The difference between 이,가/은,는 is very small. The meaning changes in the context. \\
고양이 \red{는} 집 뒤에 있다 = The cat is behind the house.\\
고양이 \red{가} 집 뒤에 있다 = The cat is behind the house.\\
\emph{는,은} have the role of being compared to something else. It means when we say there's a cat behind the house, other things are \red{not} behind the house. So the translation could change to: It is the cat that is behind the house.\\
But it's like in other languages. The difference lays more in the feeling which one to use.
\end{Korean}

\subsubsection{objectmarker}
\begin{tabular}{p{200pt}p{200pt}}
\hline
	\textcolor{brown}{object} particle 	\begin{Korean} 를, 을\end{Korean}&
	
	Vowel:\begin{Korean} \red{를}:     나를, 저를 \end{Korean}\newline
	 Consonant:\begin{Korean}\red{을}: 집을, 잭을 \end{Korean}\\
\hline

\end{tabular}

% % % % % % % % % % % % % %
%Ending with a adjective
% % % % % % % % % % % % % %

\subsection{sentence ending with an adjective}
The Structure of a sentence ending with an adjective is:\\
\impbox{\subject{subject} - \adj{adjective}}
The difference between a sentence ending with a verb and an adjective is the following. adjective cannot act on an object

\subsubsection{subjectmarker}
\begin{tabular}{p{200pt}p{200pt}}
\hline
	\textcolor{red}{subject} particle   \begin{Korean} 은, 는\end{Korean}&
	
	Vowel:\begin{Korean} \red{는} 나는, 자는 \end{Korean} \newline
	Consonant:\begin{Korean}\red{은}: 집은,잭은 \end{Korean}\\
\hline
\end{tabular}


\subsubsection{nounmarker}
\begin{tabular}{p{140pt}p{300pt}}
\hline
	\textcolor{brown}{noun} particle   \begin{Korean} 이, 가\end{Korean}&
	
	Vowel: \begin{Korean} \red{가}:    나는 자가 있다 \end{Korean} =   I have a new car \newline
	Consonant: \begin{Korean}\red{이}: 나는 펀이 있다 \end{Korean} =   I have a pen\\
\hline
\end{tabular}
\\\\There are some "irregular adjectives"\\
	\begin{Korean}이다\end{Korean}: to be  acts like an \textcolor{green}{adjective}\\
	\begin{Korean}있다\end{Korean}: to have  acts like an \textcolor{green}{adjective}\\\\
	
	


% % % % % % % % % % % % % %
%Particles
% % % % % % % % % % % % % %

\subsection{particles}

\subsubsection{Particle for place or time}
\begin{Korean}
The Particle to describe time or a place is \red{에}: I는 3pm 에 went \\
To specify the exact position: position words are in \red{front} of 에\\
학교 앞에. = in front of the school.\\
사람 뒤다. = behind the person.\\
some position words:\\
\begin{tabular}{lll}
\hline
\begin{Korean}안 = inside\end{Korean}&\begin{Korean}위 = on top\end{Korean}&\begin{Korean}밑 = below\end{Korean}\\
\begin{Korean}옆 = beside\end{Korean}&\begin{Korean}뒤 = behind\end{Korean}&\begin{Korean}앞 = in front\end{Korean}\\
\hline
\end{tabular}
\end{Korean}

\subsubsection{this and that}
\begin{Korean}
There are three words for this and that, \red{이, 저, 그 }\\
	이: something is in touching distance\newline
	그: talking from previous sentence\newline
	저: speaking about something far away\\\\
using this/that with  이다(to be): The sentence \red{begins} with this particle \\
\red{그}사람은 남자이다 = that person is a man
\end{Korean}
\subsection{To be in a location}
\begin{Korean}
있다 has another meaning than to have. It can also mean to \red{be in a location}. To recognize this meaning 에 is often used in a sentence with 있다.\\
나는 학교에 있다. = I am at school.\\
나는 학교가 있다  = I have a school\\
\end{Korean}

\subsection{Possessive Particle}
\begin{Korean}
의 is a particle that indicates that one is the owner/possessor of another object. It has to be placed \red{at the end} of the word, which is \red{possessor}.\\
저의 첵 = my book\\
나는 선생님의 차를 원하다 = I want the teacher's car.\\
그 여자의 눈은 예쁘다. = That women's eyes are beautiful.
\end{Korean}

\subsection{good, like}
\begin{Korean}
좋다 is an \adj{adjective} that means \red{good}.\\
이 음식은 좋다 = this food is good.
그 선생님은 좋다 = that teacher is good.\\
\\
There is a \ver{verb} 좋아하다 which meaning is \red{like}.\\
나는 이 음식을 좋아하다 = I like this food.\\
나는 그 선생님을 좋아하다 = I like that teacher.
\end{Korean} 


