\section{Basic Korean}
\red{Three Basic rules}\\
\impbox{\begin{enumerate}
\item \red{Every Korean sentence must end in either a verb or an adjective}
\item Every Korean verb and adjective ends with \begin{Korean}다\end{Korean} in the dictionary form
\item A lot of verbs and adjectives end with \begin{Korean}하다\end{Korean}. To get the noun of these words simply \begin{Korean}하다\end{Korean} has to be \red{removed}.\\
e.g.: \begin{Korean}
행복하다 = happy$\qquad$ 행복    = happyness
\end{Korean}
\end{enumerate}}
% % % % % % % % % % % % % %
%Ending with a verb
% % % % % % % % % % % % % %
\subsection{sentence ending with a verb}
The Structure of a sentence ending with a verb is: \\
\impbox{\textcolor{red}{Subject} - \textcolor{brown}{object} - \textcolor{blue}{verb}}

\subsubsection{subject marker}

\begin{tabular}{p{200pt}p{200pt}}
\hline
	 \textcolor{red}{subject} particle  	\begin{Korean} 은, 는\end{Korean}&
	
	Vowel:\begin{Korean} \red{는}: 나는, 자는 \end{Korean} 	\newline
	Consonant:\begin{Korean}\red{은}: 집은,잭은 \end{Korean}\\
	
\hline
	 \textcolor{red}{subject} paticle  	\begin{Korean} 이, 가\end{Korean}&
	
	Vowel:\begin{Korean} \red{가}: 고양이가 집 뒤에 있다 \end{Korean} 	\newline
	Consonant:\begin{Korean}\red{이}: 젝이 집 뒤에 있다 \end{Korean}\\
\hline
\end{tabular}\\
\\\\
\begin{Korean}The difference between 이,가/은,는 is very small. The meaning changes in the context. \\
고양이 \red{는} 집 뒤에 있다 = The cat is behind the house.\\
고양이 \red{가} 집 뒤에 있다 = The cat is behind the house.\\
\emph{는,은} have the role of being compared to something else. It means when we say there's a cat behind the house, other things are \red{not} behind the house. So the translation could change to: It is the cat that is behind the house.\\
But it's like in other languages. The difference lays more in the feeling which one to use.
\end{Korean}

\subsubsection{objectmarker}
\begin{tabular}{p{200pt}p{200pt}}
\hline
	\textcolor{brown}{object} particle 	\begin{Korean} 를, 을\end{Korean}&
	
	Vowel:\begin{Korean} \red{를}:     나를, 저를 \end{Korean}\newline
	 Consonant:\begin{Korean}\red{을}: 집을, 잭을 \end{Korean}\\
\hline

\end{tabular}

% % % % % % % % % % % % % %
%Ending with a adjective
% % % % % % % % % % % % % %

\subsection{sentence ending with an adjective}
The Structure of a sentence ending with an adjective is:\\
\impbox{\textcolor{red}{Subject} - \textcolor{green}{adjective}}

\subsubsection{subjectmarker}
\begin{tabular}{p{200pt}p{200pt}}
\hline
	\textcolor{red}{subject} particle   \begin{Korean} 은, 는\end{Korean}&
	
	Vowel:\begin{Korean} \red{는} 나는, 자는 \end{Korean} \newline
	Consonant:\begin{Korean}\red{은}: 집은,잭은 \end{Korean}\\
\hline
\end{tabular}


\subsubsection{objectmarker}
\begin{tabular}{p{140pt}p{300pt}}
\hline
	\textcolor{brown}{object} particle   \begin{Korean} 이, 가\end{Korean}&
	
	Vowel: \begin{Korean} \red{가}:    나는 자가 있다 \end{Korean} =   I have a new car \newline
	Consonant: \begin{Korean}\red{이}: 나는 펀이 있다 \end{Korean} =   I have a pen\\
\hline
\end{tabular}
\\\\There are some "irregular adjectives"\\
	\begin{Korean}이다\end{Korean}: to be  is an \textcolor{green}{adjective}\\
	\begin{Korean}있다\end{Korean}: to have  acts like an \textcolor{green}{adjective}\\\\
	
	


% % % % % % % % % % % % % %
%Particles
% % % % % % % % % % % % % %

\subsection{particles}

\begin{tabular}{p{200pt}p{200pt}}
\hline
	Particle for place or time \begin{Korean}에\end{Korean} &
	I\begin{Korean}는\end{Korean} 3pm \begin{Korean}에\end{Korean} went\\

	specify exact position: position words in front of \begin{Korean}에\end{Korean} &
	\begin{Korean}학교 앞에\end{Korean} = in front of the school\newline
	\begin{Korean}사람 뒤다\end{Korean} = behind the person\\
\hline
	This and that: \begin{Korean}이, 저, 그\end{Korean}&
	\begin{Korean}이\end{Korean}: something is in touching distance\newline
	\begin{Korean}그\end{Korean}: talking from previous sentence\newline
	\begin{Korean}저\end{Korean}: speaking about something far away\\
\hline
	 \begin{Korean}있다\end{Korean}: to be in a location&
	 \begin{Korean}나는 학교에 있다\end{Korean} = I am at school\newline
	 \begin{Korean}나는 학교가 있다\end{Korean} = I have a school\\
\hline		

\end{tabular}

